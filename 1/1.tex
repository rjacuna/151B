\documentclass{article}
\usepackage{fontspec}
\usepackage{xcolor}
%\usepackage{sagetex}

\usepackage{euler}
\usepackage{amsmath}
\usepackage{unicode-math}


\usepackage[makeroom]{cancel}
\usepackage{ulem}

\setlength\parindent{0em}
\setlength\parskip{0.618em}
\usepackage[a4paper,lmargin=1in,rmargin=1in,tmargin=1in,bmargin=1in]{geometry}

\setmainfont[Mapping=tex-text]{Helvetica Neue LT Std 45 Light}

\newcommand\N{\mathbb{N}}
\newcommand\Z{\mathbb{Z}}
\newcommand\R{\mathbb{R}}
\newcommand\C{\mathbb{C}}
\newcommand\A{\mathbb{A}}

\usepackage{soul}
\begin{document}

\begin{center}
  151B --- Homework 1

  Ricardo J. Acuna

  (862079740)
\end{center}\vspace{1.618em}

\paragraph{1} Find an alternative proof to the Bolzano–Weierstrass Theorem as follows: Let $B$ be the
set of points such that $f(t) > c$. Show that the greatest lower bound of $B$ exists. Let $x$ be
the glb of $B$. Prove that $f(x) = c$.

\uwave{pf.}

Given $[a,b]\subset \R$, and a continuous map $f$
defined on $[a,b]$. Want to show,
\[f(a)<f(b) \implies \forall c\in (f(a),f(b)):
\exists x\in(a,b): f(x) = c.\]

$\forall c \in(f(a),f(b)),$ define $ B_c:=\{t \in [a,b]| f(t) > c\}.$

$f(b) \in B_c \implies B_c \neq \emptyset .$

$f^{-1}(B_c)\subset [a,b] \implies B_c = f(f^{-1}(B_c)) \subset
f([a,b]) .$

$f$ is continuous $\implies f([a,b])$ is an interval.

Now, $B_c$ is a bounded, not empty set in $\R$. By the completeness of
$\R$, the $\inf B_c$ exists.

Let $x = \inf B_c$. Want to show by contradiction that $f(x) = c$,

Ass. $f(x) > c ,$

Put $\epsilon = f(x) - c \geq 0$. By the continuity of $f$,

\[\exists \delta > 0: |t - x | < \delta \implies |f(t) - f(x)| <
  \epsilon.\]

$\implies |f(t) - f(x)| < f(x) - c.$

$\implies -(f(x) - c) < f(t) - f(x) < f(x) - c.$

$\implies c -f(x) < f(t) - f(x).$

$\implies c < f(t).$

$\implies c < f(t_0).$ whenever $ x-\delta < t_0 <x$

$\implies x$ is a not a lower bound of $B_c.$

Ass. $f(x) < c,$

Put $\epsilon = c - f(x) > 0$. By the continuity of $f$,

\[\exists \delta > 0: |t - x | < \delta \implies |f(t) - f(x)| <
  \epsilon.\]

$\implies  |f(t) - f(x)| < c - f(x).$

$\implies  f(t) - f(x) < c - f(x).$

$\implies  f(t) < c.$

$\implies t \not \in B_c$ and $t\in (x-\delta , x +\delta).$

For some $t_1$, $x+\delta > t_1 > x$. $t_1$ is a bigger bound than
$x$.

$\implies x$ is not the greatest lower bound of $B_c.$

So, by contradiction $f(x) = c\quad ∎$

\paragraph{2} Let $E = {x_n }$ be a countable subset of $[a, b]$ and
${c_n}$ be a sequence of positive numbers such that
$c_n$ converge and is finite. Define

\[f (x) =\sum_{x_n < x} c_n.\]

Prove the following:

(1) f is monotonically increasing on $[a, b]$.

(2) f is discontinuous at every points of E with
$f (x_n +) − f (x_n −) = c_n $.

(3) f is continuous at every other point of $(a, b)$.

\uwave{pf. of (a)}

We want to show $x<y \implies f(x) < f(y).$

Let $x,y \in [a,b]$, without loss of generality ass. $x<y$.

By definition, $f(x) =\sum_{x_n < x} c_n,$ and $f(y) =\sum_{x_n < y}
c_n.$

$f(x) = \sum_{x_n < x} c_n \leq \sum_{x_n < x} c_n + \sum_{y < x} c_n
= \sum_{x_n < y} c_n = f(y)$.

We can do this because $\forall n: c_n > 0 \quad ∎$

\uwave{pf. of (b) and (c)}

Since $f$ is increasing by 4.29

$f(x+)= \sup_{a<t<x} f(t) < f(x) < \inf_{x<t<b} f(t) = f(x-)$

Since every bounded sequence has a convergent subsequence,

$\{x_n\} \subset [a,b] \implies \exists \{x_{n_k}\}$ such that
$\lim_{k\rightarrow \infty} x_{n_k} = x$, furthermore we can choose
the subsequence such that $\forall x_i \in \{x_n\} \backslash
\{x_{n_k}\}.$ Such that, $ x_i \geq x.$ Then we get a subsequence that
has a matching $c_{n_k},$ for each $x_{n_k},$ and it covers all of the
values between $a,$ and $x$, not including $x$.

So since all the values of the sequence are all smaller than $x$, and
we chose the sup, it must be that,

$f(x-) = \sum_{x_n < x} c_n = f(x).$

Similarly, we can choose a subsequence that converges to $x$ that covers
all the values between $x$, and $b$, not including $x$.

So, since the values of the sequence are all bigger than $x$, and we
chose the inf, it must be that,

$f(x+) = \sum_{x_n \leq x} c_n$.

Now, $\forall x_i \in E: f(x_i+) - f(x_i-) = \sum_{x_n \leq x_i} c_n -
\sum_{x_n < x_i} c_n = c_i$.

Which shows $f$ is discontinuous for all $x_i\in E$.

Now if $x\not\in E \implies x_n < x <x_{n+1}$ for some $n\in\N.$

$\implies x_n < x \implies x_n \neq x.$

Now,$x_n\leq x$ and $x_n\neq x \implies x_n < x$, which reduces
$f(x+)$ to $f(x)$.

$\implies f(x+)-f(x-) = f(x) - f(x) = 0 \implies f(x+) = f(x-)$

So, $f$ is continuous for all $x\not\in E. \quad ∎$
\newpage
\paragraph{3} Suppose $f$ and $g$ are defined and that $f(t)→ A$ and
$g(t)→ B$ as $t → +∞$ where $A$ and $B$
are real numbers. Prove that $(f + g)(t) → A + B$ and $(f g)(t) → AB$
as $t → +∞$.

\uwave{pf.} $f(t)→ A$ as $t → +∞$, and $g(t)→ B$ as $t → +∞$

$\implies \forall \epsilon>0: \exists M,N\in\N: m>M,$ and $n> N$ and
sequences $\{t_n\}$ and $\{t_k\}$ such that,


$|g(t_m) - B |<\frac{\epsilon}{2}$ and $|f(t_n)-A| <
\frac{\epsilon}{2}$

Put $K = \max\{N,M\}, k>K \implies |f(t_k) +g(t_k) -(A + B)| <
|f(t_k)-A| +|g(t_k)-B| < \epsilon$.

So, $(f+g)(t) \rightarrow A+B$ as
$t\rightarrow +\infty$

$\epsilon$ was arbitrary so put,

$|g(t_k) - B |<\frac{\epsilon}{2|B|}$ and $|f(t_k)-A| <
\frac{\epsilon}{2|A|}$

We can see,

$|f(t_k)g(t_k) - AB|= |f(t_k)g(t_k) -f(t_k) B +f(t_k) B - AB| <
|f(t_k)g(t_k) -f(t_k)B |+|f(t_k) B - AB| = |f(t_k)(g(t_k)-B)|
+|B(f(t_k)-A)|=  |f(t_k)||g(t_k)-B|
+|B||f(t_k)-A|\leq |f(t_k)|\frac{\epsilon}{2|A|} +|B|\frac{\epsilon}{2|B|}$

Taking the limit $k \rightarrow +\infty$  gives,

$|f(t_k)g(t_k) - AB| < |A|\frac{\epsilon}{2|A|}
+|B|\frac{\epsilon}{2|B|} = \epsilon$

So $(fg)(t) \rightarrow AB$ as $t\rightarrow +\infty.$$\quad ∎$

\paragraph{4} We say that $f$ is one-to-one on $E$ if $x_1  \neq x_2$
implies $f (x_1 )\neq f(x_2 )$ for all $x_1 , x_2 \in E$. If $f$
is one-to-one and continuous on $[a, b]$ and $f (a) < f (b)$, prove that f is strictly increasing.
That is, $x_1 < x_2$ implies $f (x_1) < f (x_2 )$.

\uwave{pf.}

$\forall x_1,x_2\in E$.

Suppose $x_1 < x_2$ and $f(x_1) = f(x_2)$,

$f$ is one to one $\implies x_1 = x_2$, which is false, so $f(x_1)\neq
f(x_2).$

Suppose $x_1 < x_2$ and $f(x_1) > f(x_2)$,

Let $\epsilon = f(x_1)-f(x_2) > 0$

$f$ is continuous $\implies \exists \delta >0: |x_2-x_1| < \delta
\implies |f(x_2)-f(x_1)|< \epsilon$.

$\implies |f(x_2) - f(x_1)|< f(x_1) - f(x_2)$

$\implies -(f(x_1) - f(x_2)) < f(x_2) - f(x_1) $

$\implies -f(x_1) + f(x_2) < f(x_2) - f(x_1) $

$\implies 0 < 0$  which is false.

So, we arrive to a contradiction.

Therefore, $x_1 < x_2 \implies f(x_1) < f(x_2)$$\quad ∎$



\end{document}

%%% Local Variables:
%%% mode: latex
%%% TeX-master: t
%%% End:
