\documentclass{article}
\usepackage{fontspec}
\usepackage{xcolor}
%\usepackage{sagetex}

\usepackage{euler}
\usepackage{amsmath}
\usepackage{amssymb}
\usepackage{unicode-math}


\usepackage[makeroom]{cancel}
\usepackage{ulem}

\setlength\parindent{0em}
\setlength\parskip{0.618em}
\usepackage[a4paper,lmargin=1in,rmargin=1in,tmargin=1in,bmargin=1in]{geometry}

\setmainfont[Mapping=tex-text]{Helvetica Neue LT Std 45 Light}

\newcommand\N{\mathbb{N}}
\newcommand\Z{\mathbb{Z}}
\newcommand\R{\mathbb{R}}
\newcommand\C{\mathbb{C}}
\newcommand\A{\mathbb{A}}

\usepackage{soul}
\begin{document}

\begin{center}
  151B --- Homework 2

  Ricardo J. Acuna

  (862079740)
\end{center}\vspace{1.618em}

\paragraph{1} Let $f$ be defined for all real $x$, and suppose that

\[|f(x)-f(y)|<(x-y)^2\]

for all real $x$ and $y$. Prove that $f$ is constant.

\uwave{pf.}

$|f(x)-f(y)| < (x-y)^2 \iff -(x-y)^2 < f(x)-f(y)< (x-y)^2$.

Case 1: $x-y > 0$.
$-(x-y) < \frac{f(x)-f(y)}{x-y} < x-y$

Case 2: $x-y < 0$.
$-(x-y) > \frac{f(x)-f(y)}{x-y} > x-y$

In both cases, taking the limit as $x\rightarrow y\implies  x-y
\rightarrow 0$.

We squeeze the difference quotient in between $0$ and $0$.

So $f^\prime(x) = 0 \implies f$ is constant
$\blacksquare$

\newpage

\paragraph{2} If

\[C_0 +\frac{C_1}{2}+\cdots + \frac{C_{n-1}}{n} +\frac{C_n}{n+1} = 0\]

for real constants $C_0,\cdots,C_n$, prove that the equation
\[C_0 +C_1 x+\cdots+C_n x^n = 0\]

has at least one real root between $0$ and $1$.

\uwave{pf.}

\[C_0 +\frac{C_1}{2}+\cdots + \frac{C_{n-1}}{n} +\frac{C_n}{n+1} = 0 \implies C_0 = -(\frac{C_1}{2}+\cdots + \frac{C_{n-1}}{n}
  +\frac{C_n}{n+1})\]

Let $P(x) = C_0 +C_1 x+\cdots+C_n x^n$

If $C_0 <0$, then $P(0) = C_0 < 0$
and
$P(1) = C_0 + C_1+ \cdots + C_n = -(\frac{C_1}{2}+\cdots + \frac{C_{n-1}}{n}
+\frac{C_n}{n+1})+ C_1+ \cdots + C_n$


$C_0<0 \implies -C_0 > 0 \implies (\frac{C_1}{2}+\cdots + \frac{C_{n-1}}{n}
+\frac{C_n}{n+1}) > 0$


$\frac{C_i}{i+1} < C_i \forall i \in \N$ and $C_i \in \R$. And
$\forall a,b,c \in \R a<b \implies a + c < b +c$.

Repeated applications of the previous line yield,
$P(1) > 0$

$P$ is a polynomial, therefore it is continuous.

Since $P$ satisfies the hypotheses of the intermediate value theorem,

\[\exists x\in(0,1): P(x)=0\]

Rinse and repeat, $C_0 > 0 \implies P(0) > 0$ and $P(1)<0 \implies
\exists x\in(0,1): P(x)=0$.

If $C_0 = 0,$ then $P(x) = C_1x + \cdots+ C_n x^n$

We want to find, if $P$ has a root between $0$ and $1$. That is we
want to solve,

\[0 = C_1x + \cdots+ C_n x^n\]

$x \in(0,1) \implies  x \neq 0,$ so we can divide by $x$,

\[0 = C_1 + \cdots+ C_n x^{n-1}\]

Now, we look at the zeroes of $Q(x) = C_1 + \cdots+ C_n x^{n-1}$

Then, by the same argument as before $C_1<0$ and $C_1 >0$, imply there
is a root of $Q(x)$, between $0$ and $1$, which in turn implies there is
a root of $xQ(x) = P(x)$.

Now, since there are finitely many $C_i$, there are finitely many
applications of this procedure which yield a real zero in
between $0$ and $1$.

Note that, if $C_0,\cdots, C_{n-1}$ are all zero. Then,

\[C_0 +\frac{C_1}{2}+\cdots + \frac{C_{n-1}}{n} +\frac{C_n}{n+1} = 0
  \implies C_n = 0\]

So, $P(x) = Z(x)$ the zero polynomial, which clearly has a zero
between $0$ and $1$.



So,

\[C_0 +C_1 x+\cdots+C_n x^n = 0\]

has a real root between $0$ and $1$
$\blacksquare$

\paragraph{3} Suppose $f$ is continuous for $x ≥ 0$, $f^\prime(x)$ exists for $x > 0$, $f(0) = 0$, and $f^\prime$
is monotonically
increasing. Define, for $x > 0$
\[g(x) = \frac{f(x)}{x}\].

Prove that g is monotonically increasing.

\uwave{pf.}

Compute $g^\prime(x) = \frac{xf^\prime(x) -f(x)}{x^2}$ by the
quotient rule. Which exists because $f^\prime$ exists for $x>0$.

So, $g$ is differentiable on $(-\infty,0)$.

Want to show $g^\prime(x) =  \frac{xf^\prime(x) -f(x)}{x^2} \geq 0 \iff
xf^\prime(x) - f(x) \geq 0$ since $x^2>0 \forall x \neq 0.$


$f^\prime$ is increasing, so $x < y \implies f^\prime (x) \leqf^\prime(y)$

$f$ is defined on $[0,\infty]\implies f$ is defined on $[0,x].$

$f$ is differentiable on $(0,\infty) \implies f$ is differentiable on
$(0,x)$.

By the mean value theorem, $\exists t\in(0,x): f(x) -f(0) =
(x-0)f^\prime(t)$

$f(0) = 0 \implies f(x) = xf^\prime(t)$

$t\in(0,x) \implies t<x \implies f^\prime(t)<f^\prime(x)$

$\implies f(x) =xf^\prime(t) < xf^\prime(x)$ $\blacksquare$






\paragraph{4} If $f$ is differentiable in $(a,b)$ and $f^\prime(x) >
0$. Prove that $f$ is strictly increasing in $(a,b)$. Let g be its inverse function. Prove that g is differentiable and that,

\[ g^\prime(f(x)) = \frac{1}{f^\prime(x)} \]

\uwave{pf.}

$f^\prime(x) > 0 \implies $ f is strictly increasing on $(a,b)$.

Fix $s \in \text{dom}(g)$. & Let $\phi(t) = \frac{g(t) - g(s)}{t-s}.$

Since, dom$(g) = f((a,b)),$ $\exists x,y \in (a,b): f(x)= t$ and
$f(y)=s$.

Now, $\phi(f(x)) = \frac{g(f(x)) - g(f(y))}{f(x)-f(y)} = \frac{x -
  y}{f(x)-f(y)} = \frac{1}{\frac{f(x)-f(y)}{x-y}}$.

Since $f$ is differentiable it is continuous, and since $f$ has an
inverse it is $1\rightarrow 1$. So it follows $g$ is continuous.

$ \lim_{s\rightarrow t} \phi(t) = g^\prime(f(x)) =\lim_{f(x)\rightarrow f(y)}
\phi(f(x)) \stackrel{g\text{ is continuous}}{=} \lim_{x\rightarrow y}
\phi(f(x)) = \frac{1}{\lim{x\rightarrow y} \frac{f(x)-f(y)}{x-y}}
\stackrel{\text{ definition of} f^\prime}{=} \frac{1}{f^\prime(x)}$
$\blacksquare$

\end{document}

%%% Local Variables:
%%% mode: latex
%%% TeX-master: t
%%% End:
