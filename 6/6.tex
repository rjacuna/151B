\documentclass{article}
\usepackage{fontspec}
\usepackage{xcolor}
%\usepackage{sagetex}

\usepackage{euler}
\usepackage{amsmath}
\usepackage{amssymb}
\usepackage{unicode-math}


\usepackage[makeroom]{cancel}
\usepackage{ulem}

\setlength\parindent{0em}
\setlength\parskip{0.618em}
\usepackage[a4paper,lmargin=1in,rmargin=1in,tmargin=1in,bmargin=1in]{geometry}

\setmainfont[Mapping=tex-text]{Helvetica Neue LT Std 45 Light}

\newcommand\N{\mathbb{N}}
\newcommand\Z{\mathbb{Z}}
\newcommand\Q{\mathbb{Q}}
\newcommand\R{\mathbb{R}}
\newcommand\C{\mathbb{C}}
\newcommand\A{\mathbb{A}}

\usepackage{soul}
\begin{document}

\begin{center}
  151B --- 6

  Ricardo J. Acuna

  (862079740)
\end{center}\vspace{1.618em}

\paragraph{1} Consider the $f(x)$ on $[0, 1]$ such that $f(x) = 0$ if
$x = \frac{1}{2^n}$ for some positive integer $n$ and $f(x) = 1$ otherwise. Prove that $f(x)$ is integrable and compute its
integral.

\uwave{pf.}

\begin{align*}
\left\{\frac{1}{2^n}\right\}_{n=1}^\infty &\subset \bigcup_{n = 1}^\infty \left(\frac{1}{2^n}
                                            -\frac{1}{2^{n+2}},
                                            \frac{1}{2^n}+\frac{1}{2^{n+2}}\right)\\
                                           &=  \bigcup_{n = 1}^\infty
                                             \left(\frac{2^{n+2} -
                                             2^n}{2^{2n+2}},
                                             \frac{2^{n+2} +
                                             2^n}{2^{2n+2}}\right)\\
                                          &=  \bigcup_{n = 1}^\infty
                                             \left(\frac{2^{n}(4-1)
                                             }{2^{2n+2}},
                                             \frac{2^{n}(4+
                                            1)}{2^{2n+2}}\right)\\
                                          &=  \bigcup_{n = 1}^\infty
                                            \left(\frac{3
                                             }{2^{n+2}},
                                            \frac{5}{2^{n+2}}\right)\\
                                          &=  \bigcup_{n = 1}^{N-1}
                                            \left(\frac{3
                                             }{2^{n+2}},
                                            \frac{5}{2^{n+2}}\right)
                                            \bigcup \bigcup_{n = N}^\infty
                                            \left(\frac{3
                                             }{2^{n+2}},
                                            \frac{5}{2^{n+2}}\right)\\
                                          &\subset  \bigcup_{n = 1}^{N-1}
                                            \left(\frac{3
                                             }{2^{n+2}},
                                            \frac{5}{2^{n+2}}\right)
                                            \bigcup
                                            \left[  0,
                                            \frac{5}{2^{N+2}}\right)
\end{align*}

$f$ is bounded, so put $M = \sup f = 1$ on $[0,1]$.

Then Let $P = \{0,\frac{5}{2^{N+2}},\frac{1}{2^n}- \frac{5}{2^{N+2}},
\frac{1}{2^n}+ \frac{5}{2^{N+2}}, 1\}_{n=1}^{N-1}$.

$\frac{1}{2^n} \in \left(\frac{1}{2^n}- \frac{5}{2^{N+2}},
\frac{1}{2^n}+ \frac{5}{2^{N+2}}\right) \implies \varDelta x_n =
\frac{1}{2^n}+ \frac{5}{2^{N+2}} -\left(\frac{1}{2^n}-
  \frac{5}{2^{N+2}}\right) = \frac{5}{2^{N+1}}$


Choose $N$ so large such that, $\varDelta x_n = \frac{5}{2^{N+1}} <
\frac{\varepsilon}{MN}$.
\begin{align*}
\frac{1}{4} < \frac{1}{2} \implies \frac{5}{2^{N+2}} <
  \frac{5}{2^{N+1}} \implies \varDelta x_0 = \frac{5}{2^{N+2}} < \varepsilon
\end{align*}

$M_n = \sup f$ for $\frac{1}{2^n}+ \frac{5}{2^{N+2}}\leq x \leq
\frac{1}{2^n}+ \frac{5}{2^{N+2}}$,
and
$M_0 = \sup f$ for $0\leq x \leq \frac{1}{2^{N+2}}$.

$\sum_{n= 0}^{N-1} M_n \varDelta x_n < MN\frac{\varepsilon}{MN} = \varepsilon$

So, now we've isolated all of the discontinuities of $f$.

Refine $P$ to $P^*$ such that $\varDelta x_{n_i} = \frac{5}{2^{N+1}}$, for all
$\frac{1}{2^n} + \frac{5}{2^{N+2}}\leq x \leq \frac{1}{2^{n+1}} -
\frac{5}{2^{N+2}}$.

On each of the intervals corresponding to $\varDelta
x_{n_i}$, and $M_{n_i} = m_{n_i} = 1$.

Then $U(P,f)-L(P,f) <\varepsilon$ since the only difference is the
intervals corresponding to the $n$-index.

Now, since we have that $f\in \mathcal{R}.$

Then, we can pass to sample points. Since the integral over the
discontinuities can be made arbitrarily small it's obvious that,

$\int_0^1 f dx = \int_0^1 dx = 1$

$\blacksquare$

\newpage
\paragraph{2} Is $f$ as in the problem integrable?
\uwave{pf.}
$f$ is continuous at $s_n$, so $f$ satisfies the requirements of
(6.15) at each $s_n$

(6.15) $\implies \int f dI(x-s_n) = f(s_n)$

(6.12)$\implies \int f d\sum_{n=1}^N c_nI(x-s_n) = \sum_{n=1}^Nc_nf(s_n)$

$\alpha_1(x) = \sum_{n=1}^N c_nI(x-s_n)$ and $\alpha_2(x) =
\sum_{n=N+1}^\infty c_nI(x-s_n)$

Since, $\sum_{n=1}^\infty c_n$ converges $\implies \sum_{n=1}^\infty
c_nI(x-s_n)$ converges.

$\implies \sum_{n=N}^\infty
c_nI(x-s_n) <\varepsilon$ for sufficiently large $N$.

Let $P$ be any partition of $[a,b]$.

$f$ is bounded on $(a,b) \implies |f|\leq M$ $\implies \sum_{n=N}^\infty
f(s_n)\varDelta \alpha_{2}_i <M\varepsilon$

Let $\alpha(x) = \alpha_1(x)+\alpha_2(x) \implies |\int fd\alpha -\sum_{n=1}^N
f(s_n)|< M\varepsilon$

Take the limit as $N\rightarrow \infty$ and the result follows.

$\blacksquare$

\paragraph{3} As in the problem.

\uwave{pf.}

Since $\phi$ is a continuous one to one map from $[a,b]$ onto $[c,d]$, and $\phi(c)
a$. $\phi$ is surjective. By that $\phi([c,d]) = [a,b]$. Since $\phi$
is continuous and surjective it is a homeomorphism so it admits a
continuous inverse.$\gamma_2 = \gamma_1\circ \phi$

$(\implies)$ Ass. $\gamma_2$ is rectifiable $\Lambda(\gamma_2) <\infty$

So,

\paragraph{4}

By hint:
\begin{align*}
g(t_i)\varDelta x_i = G(x_i) - G(x_{i-1}) &\implies
(\alpha(x_i) - \alpha(x_{i-1}))g(t_i)\varDelta x_i = (\alpha(x_i) -
                                            \alpha(x_{i-1}))(G(x_i) -
                                            G(x_{i-1}))\\
                                            &\implies \alpha(x_i)g(t_i)\varDelta x_i - \alpha(x_{i-1})g(t_i)\varDelta x_i = G(x_i)(\alpha(x_i) -
                                            \alpha(x_{i-1})) - G(x_{i-1})(\alpha(x_i) -
                                            \alpha(x_{i-1}))
\end{align*}
\end{document}


%%% Local Variables:
%%% mode: latex
%%% TeX-master: t
%%% End:
