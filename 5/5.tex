\documentclass{article}
\usepackage{fontspec}
\usepackage{xcolor}
%\usepackage{sagetex}

\usepackage{euler}
\usepackage{amsmath}
\usepackage{amssymb}
\usepackage{unicode-math}


\usepackage[makeroom]{cancel}
\usepackage{ulem}

\setlength\parindent{0em}
\setlength\parskip{0.618em}
\usepackage[a4paper,lmargin=1in,rmargin=1in,tmargin=1in,bmargin=1in]{geometry}

\setmainfont[Mapping=tex-text]{Helvetica Neue LT Std 45 Light}

\newcommand\N{\mathbb{N}}
\newcommand\Z{\mathbb{Z}}
\newcommand\Q{\mathbb{Q}}
\newcommand\R{\mathbb{R}}
\newcommand\C{\mathbb{C}}
\newcommand\A{\mathbb{A}}

\usepackage{soul}
\begin{document}

\begin{center}
  151B --- 5

  Ricardo J. Acuna

  (862079740)
\end{center}\vspace{1.618em}

\paragraph{1} Suppose $f$ is a bounded real function on $[a,b]$ and
$f^2$ is Riemman integrable. Does it follow  that $f$ is Riemann
integrable? does the answer change if we assume that $f^3$ is Riemann integrable?

\uwave{pf.}

Consider,
\[f(x) = \begin{cases} 1\quad\quad  x \in \Q \\ -1\quad
    \text{else} \end{cases}\]

$f\not\in \mathcal{R} \impliedby$ $\int_{\underbar{}} f dx \neq
\int^{\bar{} } f dx \impliedby \forall$ partitions $P$ of $[a,b]$, $U(P,f)=
1$, and $L(P,f)= -1$.


Notice, $f^2 \equiv 1$ $\implies f^2 \in \mathcal{R}$, since constant
functions are integrable.

Therefore, $f$ is a counterexample to the assumptions of the question.


When we exchange $f^2$ by $f^3$. We have the statement,
\[f\text{ is a bounded real function on }[a,b]\text{ and
  }f^3\in\mathscr{R} \implies f\in \mathscr{R} \]

Since the cube root is continuous on $\R$ it is integrable, so the
composition $(f^3)^{1/3} = f \in\mathscr{R}$.

$\blacksquare$

\paragraph{2} That $E$ be the Cantor set constructed in
Sec. $2.44$. Let $f$ be a bounded real function on $[0,1]$ which is
continuous at every point outside of $E$. Prove that $f$ is Riemann
integrable.
\textbf{Hint} Cover $E$ by finitely many segments whose total length can be made as small as
desired. Proceed as in Theorem $6.10$.

\uwave{pf. }

$$E = \bigcap_{n=1}^\infty \bigcap_{k = 0}^{3^{(n-1)} -1} \left(
  \left[ 0, \frac{3k+1}{3^n} \right] \cup \left[ \frac{3k+2}{3^n}, 1\right]\right)$$

We can see that $$E \subset K = \left\{\left(\frac{4- 3n}{3^n},
    \frac{3n+4}{3^n}\right)\right\}_{k=1}^\infty$$.

$E$ is compact so, we can choose a finite subcover $K_k$ of $E$ such
that the total length of $K_k <\varepsilon.$

Now, $L = [0,1]\K_k$ is compact, so since $f$ is continuous on $L$,
$f$ is uniformly continuous on $L$, so $f$ is integrable on each of
the sub-intervals of $L$.

Since, $f$ is bounded, if follows that $f < M < \infty$. Therefore on
$K_k$, the $U(P,f) -L(P,f) < M \varepsilon$, for each of the closed
sets that are the left and right endpoints of each of the open sets
that make up $K_k$. Therefore, $f$ is integrable on $K_k,$ and if
$K_k$ is composed of $l$ many sets, the total value of the sums of the
integrals is less than $lM\varepsilon.$

So adding up everything gives us that $f\in \mathcal{R}$ on $[0,1]$,
as $lM\varepsilon$ is a constant multiple of $\varepsilon$ and
$\varepsilon$ was arbitrary.

$\blacksquare$

\paragraph{3} Suppose $f$ is Riemann integrable on $[a,b]$ for all
$b>a$ where $a$ is fixed. Define
\[\int_a^\infty f(x)dx = \lim_{b\rightarrow \infty} \int_a^b f(x)dx\]
if the limit exists and is finite.
Assume that $f(x) \geq 0$ and $f$ decreases monotonically  on
$[1,\infty)$. Prove that
\[\int_1^\infty f(x)dx\]
Converges if and only if
\[\sum_{n=1}^\infty f(n)\]
Converges.

\uwave{pf.}

$(\implies)$ Ass. $\int_{1}^{\infty}f dx $ converges.

$ f \geq 0\implies 0 \leq \int_{1}^\infty f dx = C \in\R$

$\implies 0\leq \int_{1}^N f dx < C$

Put, $P = \{1,2,3,\dots, N\}$ a partition of $[1,N] \implies \varDelta x_n =1$

$\implies 0\leq \sum_{n=1}^N f(n) = \sum_{n=1}^N f(n)\varDelta x_n = U(P,f) < \int_{1}^N f dx < C$.

So, $\sum_{n=1}^N f(n)$ converges for every finite $N$.

$(\impliedby)$ Ass. $\sum_{n=1}^\infty f(n)$ converges.

$f\geq 0 \implies 0 < \int_{1}^\infty f dx$

Consider, again the same partition $P$ of $[1,N]$

$\sum_{n = 1}^\infty f(n) = D \in \R \implies \sum_{n=1}^{N}
f(n)\varDelta x_n + \sum_{n=N+1}^\infty f(n)\varDelta x_n = D$

Now, for each $N \in \N, \sum_{n=1}^N f(n)\varDelta x_n = U(P,f)$

$\implies 0 < \int_1^Nf(x) dx < U(P,f) < D$.

We can choose $N$ sufficiently large such that, $\sum_{n =
  N+1}^{\infty} f(n)\varDelta x_n <\varepsilon$

For the same reasons we have $\int_{N}^\infty f dx <\varepsilon$


Combining we get $0 < \int_1^N f dx+ \int_N^\infty fdx = \int_1^\infty
f dx < D+\varepsilon$.

Since $\varepsilon$, was arbitrary we have that $\int_1^\infty f dx$ converges.

$\blacksquare$


\paragraph{4} $f\in \mathscr{R} (\alpha_1)$ and
$f\in\mathscr{R}(\alpha_2)$, then
$f\in\mathscr{R}(\alpha_1+\alpha_2)$ and \[\int_a^b
  fd(\alpha_1+\alpha_2) = \int_a^b
  fd(\alpha_1) +\int_a^b
  fd(\alpha_2)  \]

\uwave{pf.}

Let $\alpha = \alpha_1 +\alpha_2$, and $P=\{x_0,\dots,x_n\}$ be any partition of
$[a,b].$
\begin{align*}
\varDelta\alpha_i &= \alpha_i(x_i)-\alpha(x_{i-1}) \\&=
\alpha_{1i}(x_i)+\alpha_{2i}(x_i)-(\alpha_{1i}(x_{i-1})+\alpha_{2i}(x_{i-1}))\\
&= \alpha_{1i}(x_i)-\alpha_{1i}(x_{i-1}) +
\alpha_{2i}(x_i)-\alpha_{2i}(x_{i-1})\\ &= \varDelta\alpha_{1i}+\varDelta\alpha_{2i}\end{align*}

So,
$$L(P,f,\alpha_1) = \sum_{i=1}^n m_i \varDelta\alpha_{1i}\text{ and }L(P,f,\alpha_2) = \sum_{i=1}^n m_i \varDelta\alpha_{2i}$$

Notice,
\begin{align*}
  L(P,f,\alpha) &= \sum_{i=1}^n m_i \varDelta\alpha_i \\
                &= \sum_{i=1}^n m_i (\varDelta\alpha_{1i} +
                  \varDelta\alpha_{2i})\\
                &= \sum_{i=1}^n m_i \varDelta\alpha_{1i} + m_i
                  \varDelta\alpha_{2i}\\
                &= \sum_{i=1}^n m_i \varDelta\alpha_{1i} +
                  \sum_{i=1}^n m_i
                  \varDelta\alpha_{2i}\\
                &= L(P,f,\alpha_1) + L(P,f,\alpha_2)
\end{align*}

Similarly, $$U(P,f,\alpha) = U(P,f,\alpha_1) + U(P,f,\alpha_2)$$

Since $f\in \mathcal{R}(\alpha_1) \implies \forall \varepsilon >0,\enskip
\exists\text{ a partition }P_1$ of $[a,b]:$

$$U(P_1,f,\alpha_1) -L(P_1,f,\alpha_1) <\frac{\varepsilon}{2}$$

Also $f\in \mathcal{R}(\alpha_2) \implies \forall \varepsilon
>0,\enskip \exists\text{ a partition } P_2$ of $[a,b]:$

$$U(P_2,f,\alpha_2) -L(P_2,f,\alpha_2) <\frac{\varepsilon}{2}$$


Since passing to their common refinement $P = P_1 \cup P_2$ maintains
the inequalities we can write,

$$U(P,f,\alpha_1) -L(P,f,\alpha_1) <\frac{\varepsilon}{2}\text{ and }U(P,f,\alpha_2) -L(P,f,\alpha_2) <\frac{\varepsilon}{2}$$

Adding the preceding inequalities yields,

$$U(P,f,\alpha_1) + U(P,f,\alpha_2) -\left(  L(P,f,\alpha_1)
  +L(P,f,\alpha_2)\right) <\varepsilon $$

So by the second and third equalities we have that,

$$U(P,f,\alpha)  -  L(P,f,\alpha) < \varepsilon \implies f\in \mathcal{R}(\alpha_1+\alpha_2)$$

Since $U(P,f,\alpha) = U(P,f,\alpha_1)+U(P,f,\alpha_2)$ taking the
infimum over all $P$, we get that $\bar{\int} f\d(\alpha_1+\alpha_2) = \bar{\int}
f d\alpha_1 + \bar{\int} f d\alpha_2$. Similarly, taking the equality
of the lower Riemann sums and applying the supremum over all $P$
gives us that $\underbar{\int} f\d(\alpha_1+\alpha_2) = \underbar{\int}
f d\alpha_1 + \underbar{\int} f d\alpha_2$. Since $f\in \mathcal{R}{\alpha_1}$,
and $f\in\mathcal{R}(\alpha_2)$, it follows that those upper and the
lower Riemann integrals are equal respectively. So,

\[\int_a^b
  fd(\alpha_1+\alpha_2) = \int_a^b
  fd(\alpha_1) +\int_a^b
  fd(\alpha_2)  \]

$\blacksquare$

\end{document}


%%% Local Variables:
%%% mode: latex
%%% TeX-master: t
%%% End:
